\documentclass{article}

% if you need to pass options to natbib, use, e.g.:
%     \PassOptionsToPackage{numbers, compress}{natbib}
% before loading neurips_2025

% The authors should use one of these tracks.
% Before accepting by the NeurIPS conference, select one of the options below.
% 0. "default" for submission
 \usepackage{neurips_2025}
% the "default" option is equal to the "main" option, which is used for the Main Track with double-blind reviewing.
% 1. "main" option is used for the Main Track
%  \usepackage[main]{neurips_2025}
% 2. "position" option is used for the Position Paper Track
%  \usepackage[position]{neurips_2025}
% 3. "dandb" option is used for the Datasets & Benchmarks Track
 % \usepackage[dandb]{neurips_2025}
% 4. "creativeai" option is used for the Creative AI Track
%  \usepackage[creativeai]{neurips_2025}
% 5. "sglblindworkshop" option is used for the Workshop with single-blind reviewing
 % \usepackage[sglblindworkshop]{neurips_2025}
% 6. "dblblindworkshop" option is used for the Workshop with double-blind reviewing
%  \usepackage[dblblindworkshop]{neurips_2025}

% After being accepted, the authors should add "final" behind the track to compile a camera-ready version.
% 1. Main Track
 % \usepackage[main, final]{neurips_2025}
% 2. Position Paper Track
%  \usepackage[position, final]{neurips_2025}
% 3. Datasets & Benchmarks Track
 % \usepackage[dandb, final]{neurips_2025}
% 4. Creative AI Track
%  \usepackage[creativeai, final]{neurips_2025}
% 5. Workshop with single-blind reviewing
%  \usepackage[sglblindworkshop, final]{neurips_2025}
% 6. Workshop with double-blind reviewing
%  \usepackage[dblblindworkshop, final]{neurips_2025}
% Note. For the workshop paper template, both \title{} and \workshoptitle{} are required, with the former indicating the paper title shown in the title and the latter indicating the workshop title displayed in the footnote.
% For workshops (5., 6.), the authors should add the name of the workshop, "\workshoptitle" command is used to set the workshop title.
% \workshoptitle{WORKSHOP TITLE}

% "preprint" option is used for arXiv or other preprint submissions
 % \usepackage[preprint]{neurips_2025}

% to avoid loading the natbib package, add option nonatbib:
%    \usepackage[nonatbib]{neurips_2025}

\usepackage[utf8]{inputenc} % allow utf-8 input
\usepackage[T1]{fontenc}    % use 8-bit T1 fonts
\usepackage{hyperref}       % hyperlinks
\usepackage{url}            % simple URL typesetting
\usepackage{booktabs}       % professional-quality tables
\usepackage{amsfonts}       % blackboard math symbols
\usepackage{nicefrac}       % compact symbols for 1/2, etc.
\usepackage{microtype}      % microtypography
\usepackage{xcolor}         % colors

% Note. For the workshop paper template, both \title{} and \workshoptitle{} are required, with the former indicating the paper title shown in the title and the latter indicating the workshop title displayed in the footnote. 
\title{SecureClaw: Non-Bypassable, Privacy-Preserving Policy Enforcement for Agentic Tool Use and Skill Supply Chains}


% The \author macro works with any number of authors. There are two commands
% used to separate the names and addresses of multiple authors: \And and \AND.
%
% Using \And between authors leaves it to LaTeX to determine where to break the
% lines. Using \AND forces a line break at that point. So, if LaTeX puts 3 of 4
% authors names on the first line, and the last on the second line, try using
% \AND instead of \And before the third author name.


\author{%
  Anonymous Authors \\
  Paper Artifact: SecureClaw \\
}


\begin{document}


\maketitle


\begin{abstract}
SecureClaw is a system level reference monitor for agentic tool use and skill supply chains.
It reduces the trusted computing base to a small gateway and an executor, while treating the agent runtime and skills as adversarial.
To prevent bypass, SecureClaw routes all high level intents through a non bypassable \textsf{PREVIEW}\,$\rightarrow$\,\textsf{COMMIT} transaction, and a dual authorization executor that requires commit proofs from two non colluding policy servers.
To reconcile centralized policy databases with query privacy, the gateway performs fixed shape policy checks against two policy servers using two server PIR based on DPF and FSS and evaluates policy programs via two party computation over XOR shared inputs.
We implement an end to end artifact including sealed handles for secret myopia, intent routing, capsule mediation, skill ingress, and real agent integrations, and we report deterministic evidence for non bypassable effects, secret myopia, single auditor privacy under a stated leakage function, and skill confined safety under an explicit capsule mediation assumption.
\end{abstract}

% Paper body sections for SecureClaw.
% This file is intended to be \input{} from neurips_2025.tex.

\section{Introduction}
\label{sec:intro}

Agentic tool use systems combine language model planning with tool invocation and skill supply chains.
The runtime processes untrusted content and can be manipulated by prompt injection, tool parameter injection, and malicious skills.
A practitioner perspective has argued that securing an agent runtime that retains user level powers can be self defeating because the restrictions required for safety also remove the utility of the runtime.
We cite \url{https://www.aikido.dev/blog/why-trying-to-secure-openclaw-is-ridiculous} as an example of this view.

SecureClaw follows a different strategy.
It treats the runtime as adversarial and moves policy enforcement and effect authorization into a small trusted gateway and a separate executor.
The runtime retains planning utility, but it cannot directly commit external side effects without cryptographic authorization that is verified at the executor.

\section{Problem statement and scope}
\label{sec:problem}

\subsection{Setting and objective}

We study agentic tool use systems where an untrusted agent runtime invokes external tools and may install third party skills.
The untrusted runtime may be compromised by prompt injection, tool parameter injection, and supply chain attacks.
The system must enforce centralized policy while preserving query privacy against any single policy server.
The system must also ensure that no external side effect is committed unless the policy enforcement path authorizes that exact request.

The key difficulty is that existing agent platforms often treat policy as advisory.
A compromised runtime can bypass a policy check by directly calling a tool, by replaying an earlier authorization, or by exfiltrating via a different channel such as inter agent messaging.
At the same time, naive centralized policy evaluation leaks sensitive query contents to the policy service.

\subsection{System model}

The system consists of five roles.

\paragraph{Untrusted runtime $\mathcal{A}$.}
The untrusted runtime includes the agent loop, all skills, and all prompt driven logic.
An adversary controls $\mathcal{A}$ and may arbitrarily choose intents, inputs, and any payload strings.
The adversary may also attempt to send arbitrary requests directly to the executor.

\paragraph{Trusted gateway $\mathcal{G}$.}
The gateway exports a single action interface to $\mathcal{A}$ and mediates all sensitive input flows.
The gateway stores sealed handles that bind secrets to a session and a caller identity.
The gateway coordinates privacy preserving policy checks with two policy servers and produces a preview transaction that can be committed.

\paragraph{Policy servers $\mathcal{P}_0$ and $\mathcal{P}_1$.}
Each policy server stores the same logical policy databases and evaluates the same policy program.
Each policy server is honest for integrity and honest but curious for privacy.
At least one policy server does not collude with the adversary.

\paragraph{Executor $\mathcal{X}$.}
The executor is the only component that can perform external side effects.
Every effectful endpoint must enforce an acceptance predicate that requires dual commit evidence.
The executor fails closed on any verification error.

\paragraph{User $\mathcal{U}$.}
The user may confirm some actions.
User confirmation is treated as an explicit declassification capability that must be recorded in the audit log.

\subsection{Trust assumptions and non goals}

Our baseline claims rely on the following assumptions.

\paragraph{Non collusion of policy servers.}
At least one policy server is not controlled by the adversary and does not collude with the other policy server.
If both policy servers collude then single auditor privacy is not achievable in our model.

\paragraph{Trusted gateway and executor.}
The gateway and executor are the trusted computing base.
If either is compromised then the system can be subverted.
We treat this as out of scope and we minimize the trusted code paths to reduce the attack surface.

\paragraph{No kernel side channel guarantees.}
We do not claim to prevent leakage through kernel level side channels such as cache timing, scheduler effects, or shared resource contention.
We explicitly quantify and bound only the application level leakage channels in our leakage function.

\paragraph{Not a general secure computation policy engine.}
The artifact implements a policy evaluation pipeline specialized to membership checks and a fixed policy circuit.
We do not claim a universal multi party computation framework for arbitrary policies.

\subsection{Contributions as general mechanisms}

SecureClaw contributes three mechanisms that are independently useful and compose cleanly.

\paragraph{Mechanism one non bypassable execution line.}
We introduce a dual authorization execution line that separates policy evaluation from side effect execution.
Policy servers issue commit proof shares that are message authentication codes over a request bound commitment.
The executor verifies two proofs, checks freshness, rejects replays, and binds the authorization to the exact request hash.
This turns policy from advisory into a non bypassable enforcement line.

\paragraph{Mechanism two leakage aware constant shape compilation.}
We compile policy queries into a fixed shape plan.
The gateway uses a unified policy program identifier, a constant number of PIR keys per action, and constant shape microbatch mixing with padding and cover traffic.
This ensures that a single policy server transcript is simulatable from an explicit leakage function.
The same compilation principle applies to both PIR membership checks and two party policy circuit evaluation.

\paragraph{Mechanism three oblivious routing and shaping.}
We reduce intent and database selection leakage by using oblivious bundle selection and unified routing.
Policy servers observe a single logical database name and a fixed endpoint schedule.
The gateway performs batching and mixing to reduce timing distinguishability at the policy server boundary.

\subsection{Applicability conditions}

The mechanisms above apply when the deployment satisfies three conditions.

\paragraph{Single effect gate.}
All external side effects must be committed by a single executor component that can enforce acceptance of commit evidence.
If tools can directly produce effects without passing through the executor then non bypassable effects cannot be guaranteed.

\paragraph{Two policy servers with a non collusion assumption.}
Single auditor privacy requires two policy servers and a non collusion assumption.
If the deployment can tolerate query exposure then a single policy server baseline may be acceptable for performance.

\paragraph{Stable canonical binding.}
Request binding requires stable canonicalization.
The gateway and executor must compute identical request hashes for semantically equal inputs.

\section{Protocol and system design}
\label{sec:protocol}

\subsection{Interface level specification}

The gateway exports a single tool interface.
Every call has the form
\[
r \gets \langle \mathsf{intent\_id},\ \mathsf{inputs},\ \mathsf{constraints},\ \mathsf{caller} \rangle .
\]
The gateway returns an observation
\[
o \gets \langle \mathsf{status},\ \mathsf{reason\_code},\ \mathsf{summary},\ \mathsf{data},\ \mathsf{artifacts} \rangle .
\]
The artifact field may include sealed handles and commit evidence.

The executor exports effectful endpoints such as send message and skill install.
Each effectful request includes a commit object that contains two policy server commit proofs.

\subsection{Sealed handles}

A sealed handle is an opaque identifier $h$ minted by the gateway.
The handle store record contains plaintext $s$, a sensitivity label, an allowed sink set, a time to live, and bindings to a session and a caller.
The untrusted runtime receives only $h$ and non sensitive metadata.

The handle invariants are as follows.

\paragraph{Opaque secrecy.}
The gateway never returns plaintext $s$ to the untrusted runtime by default.

\paragraph{Session and caller binding.}
Any attempt to use $h$ from a different session or caller is rejected.

\paragraph{Explicit declassification.}
Declassification is an explicit operation that requires user confirmation and updates a leakage budget.

These invariants are directly exercised by unit tests in \texttt{tests/test\_mcp\_gateway.py} and \texttt{tests/test\_agentleak\_channels.py}.

\subsection{Preview and commit transaction}

For effectful intents the gateway executes a preview phase and returns a transaction identifier and commit evidence.
The executor then enforces a commit phase that must verify the evidence.

Let the request binding hash be
\[
\mathsf{ReqHash} := \mathrm{SHA256}\bigl(\mathsf{Canon}(\{\texttt{v},\texttt{intent\_id},\texttt{caller},\texttt{session},\texttt{inputs}\})\bigr)
\]
where $\mathsf{Canon}$ is a deterministic canonical JSON serialization implemented once and shared by the gateway and executor.
The artifact uses \texttt{common/canonical.py:request\_sha256\_v1}.

The preview phase performs two actions.

\paragraph{Policy privacy checks.}
The gateway queries the policy databases using two server PIR and bundles the logical database selection into a single bundle database name when bundling is enabled.

\paragraph{Policy program evaluation.}
The gateway secret shares predicate inputs and invokes two party computation with a fixed circuit identifier.

The final output of preview is a fresh action identifier and two commit proof shares.

\subsection{Pseudocode and invariants}

The following pseudocode specifies the protocol level behavior of the gateway and executor.
The artifact code is in \texttt{gateway/mcp\_server.py}, \texttt{gateway/policy\_unified.py}, and
\texttt{executor\_server/server.py}.

\paragraph{Gateway algorithm.}
\begin{verbatim}
GatewayAct(intent_id, inputs, constraints, caller, session):
  if intent_id not in allowlisted_intents:
    return DENY

  inputs_eff := handleize_sensitive_fields(inputs, caller, session)
  h := ReqHash(intent_id, caller, session, inputs_eff)

  a := fresh_action_id()
  pir_ev := pir_preview(a, inputs_eff)
  mpc_ev := mpc_preview(a, h, inputs_eff, pir_ev)

  if pir_ev invalid or mpc_ev invalid:
    return DENY

  return OK with artifacts including commit evidence for a and h
\end{verbatim}

\paragraph{Executor algorithm.}
\begin{verbatim}
ExecutorCommit(req, commit):
  inputs_eff := effectful_fields(req)
  h := ReqHash(req.intent_id, req.caller, req.session, inputs_eff)

  if not verify_dual_commit_proofs(commit, action_id=req.action_id, h=h):
    return DENY

  if replay_guard_seen(req.action_id):
    return DENY

  outs, commit_tag := reconstruct_policy_outputs(commit)
  if outs.allow_pre != 1:
    return DENY
  if outs.need_confirm == 1 and req.user_confirm != true:
    return DENY

  apply_required_sanitize_patch(outs, req)
  perform_side_effect(req)

  audit_log_append(req.action_id, h, commit_tag, outs, reason_code)
  return OK
\end{verbatim}

The key invariants are as follows.

\paragraph{Fail closed.}
Any verification failure causes the gateway or executor to return deny without committing an effect.

\paragraph{Request binding.}
The executor recomputes the request hash and rejects any mismatch.
This binds authorization to the exact session, caller, and effectful inputs.

\paragraph{Audit binding.}
Every committed effect appends an audit record that contains the request hash and the reconstructed
commit tag.
These values bind the audit trail to the dual authorization evidence.

\subsection{Executor acceptance predicate}

The executor accepts a commit request if and only if all conditions hold.

\paragraph{Dual proof verification.}
Two proofs parse and verify under distinct server identities and recognized key identifiers.

\paragraph{Freshness.}
Both proofs are within a configurable time to live window.

\paragraph{Request binding.}
Both proofs bind the same action identifier, program identifier, and request hash, and the request hash matches the executor recomputation on the received request.

\paragraph{Replay resistance.}
The action identifier has not been accepted before in the replay window, optionally backed by persistent storage.

\paragraph{Policy semantics.}
The reconstructed policy decision bits authorize the effect and enforce confirmation and sanitization requirements.

The acceptance predicate is implemented in \texttt{executor\_server/server.py:\_verify\_commit\_evidence} and is validated by executable checks in \texttt{scripts/security\_game\_nbe\_check.py} and unit tests in \texttt{tests/test\_security\_games.py}.

\subsection{Fail closed invariants}

The system fails closed at every integrity boundary.

\paragraph{Gateway boundary.}
If any policy server request fails or returns invalid evidence then the gateway returns deny.

\paragraph{Executor boundary.}
If any proof verification step fails then the executor returns deny and performs no external side effect.

\paragraph{Capsule boundary.}
If the capsule mediation contract blocks a direct bypass attempt then the attempt fails without reaching any external sink.

\section{Formal security claims and proof chain}
\label{sec:security}

We state four system properties.
The full definitions and proofs are provided in Appendix \ref{app:proofs} and the accompanying artifact documents \texttt{FORMAL\_SECURITY.md} and \texttt{appendix\_security.tex}.

\paragraph{Non bypassable effects.}
No external side effect is committed unless two valid commit proofs exist that are bound to the exact request context, satisfy freshness, and are not replayed.

\paragraph{Secret myopia.}
Sensitive plaintext is never delivered to the untrusted runtime except through explicit declassification guarded by confirmation and policy.

\paragraph{Single auditor privacy.}
For each policy server there exists a simulator that can reproduce its view from an explicit leakage function.

\paragraph{Skill confined safety.}
Under a capsule mediation contract, skills cannot access host secrets, execute arbitrary processes, or access the public network except through the gateway and executor mediation path.

\section{Leakage model and implementation mapping}
\label{sec:leakage}

Single auditor privacy is defined relative to an explicit leakage function.
We define two leakage functions.

\paragraph{Policy server leakage $L_{\mathsf{policy}}$.}
This captures what a single policy server learns from PIR and two party computation transcripts.
It includes public configuration, fixed shape schedule parameters, coarse timing buckets, and action identifiers.
It excludes the queried indices and the logical intent category when unified routing is enabled.

\paragraph{System leakage $L_{\mathsf{sys}}$.}
This captures application level leakage across AgentLeak style channels.
We represent it as a tuple of per channel leakage functions $L_{C1}$ through $L_{C7}$, covering final output, inter agent messages, tool inputs, tool outputs, memory, logs, and skill ingress.

The formal definitions, per channel decomposition, residual leakage bounds, and a code and configuration mapping are given in Appendix \ref{app:leakage} and in \texttt{LEAKAGE\_MODEL.md}.

\section{Baselines and evaluation methodology}
\label{sec:eval}

\subsection{Baseline semantics}

We evaluate four SecureClaw modes and two native runtime baselines.
Each mode corresponds to a different enforcement boundary and a different threat model.

\paragraph{SecureClaw full.}
Gateway mediation, two policy servers, and dual proof executor enforcement.

\paragraph{policy only.}
Gateway mediation and policy evaluation, without executor enforcement as an effect boundary.
This isolates the value of non bypassable effects.

\paragraph{sandbox only.}
Local sandbox restrictions without policy mediation and without dual proof enforcement.
This isolates the value of policy enforcement beyond sandboxing.

\paragraph{single server policy.}
Gateway mediation and a single policy server.
This isolates the performance and privacy tradeoff by removing single auditor privacy.

\paragraph{codex native and openclaw native.}
Native runtime guardrails with no SecureClaw components.
These baselines measure what current platforms prevent under the same benchmark cases and metric definitions.

\paragraph{Compromised runtime sanity check.}
We distinguish native runtime guardrails from a compromised runtime setting where the runtime logic intentionally leaks on attack cases.
The latter is useful only to validate that prompt based defenses cannot stop an adversarial runtime that chooses to exfiltrate.
We do not present that setting as a native baseline.
The artifact records this separation explicitly in \texttt{BASELINES\_FAIRNESS.md}.

\paragraph{Mode identifiers.}
The artifact uses the mode identifiers \texttt{mirage\_full}, \texttt{policy\_only},
\texttt{sandbox\_only}, and \texttt{single\_server\_policy} for the four SecureClaw modes.

\subsection{Fairness protocol}

We enforce a same cases comparison by generating a fixed official case manifest from the benchmark dataset with a fixed random seed.
All systems are evaluated on that manifest.
For each case we record a channel, a label indicating attack or benign, a leaked flag, a blocked flag, an allowed flag, a reason code, and a latency.

The artifact implements this protocol in \texttt{scripts/fair\_full\_compare.py} and aggregates to a single report file \texttt{artifact\_out\_compare/fair\_full\_report.json}.

\subsection{Metrics and statistical analysis}

For a system $S$ and a set of attack cases $\mathcal{D}_{\mathsf{atk}}$ we define
\[
\mathsf{attack\_leak\_rate}(S) := \frac{\#\{d \in \mathcal{D}_{\mathsf{atk}} : \mathsf{leaked}(d,S)=1\}}{|\mathcal{D}_{\mathsf{atk}}|}.
\]
We define attack block rate and benign allow rate analogously.

We compute Wilson confidence intervals with $z=1.96$ for each rate.
We compute two sided Fisher exact test values for pairwise comparisons of leakage counts and block counts against the full system.
We decompose error modes by channel, by task vertical, by attack family, and by reason code.
All statistics are implemented in \texttt{scripts/fair\_full\_stats.py}.

\section{Performance and deployability}
\label{sec:perf}

\subsection{Cost model under constant shape enforcement}

The dominant cost in the artifact is the policy server kernel for PIR and fixed circuit two party computation.
Under fixed shape scheduling the policy servers observe a fixed number of subrequests per time tick.
This induces overhead from three sources.

\paragraph{Dummy compute.}
Padding and cover traffic add dummy subrequests that perform the same kernel work as real queries.

\paragraph{Mixing and queueing.}
Microbatch mixing introduces queueing delay that depends on tick interval and batch utilization.

\paragraph{Transport overhead.}
JSON encoding, copying, and connection setup increase tail latency.
Binary framing and Unix domain sockets reduce this cost in local deployments.

The artifact provides microbenchmarks and end to end benchmarks that sweep padding and batch parameters and report throughput and latency.
The narrative and reproduction commands are in \texttt{PERFORMANCE.md}.

\subsection{Why the system is not slow by construction}

The enforcement line adds cryptographic verification and constant shape scheduling, but it does not require general purpose secure computation on large data.
The executor verification work is constant time in the size of the request and is dominated by two HMAC checks and a small replay table lookup.
The policy server kernels are amenable to batch evaluation, vectorized inner products, and multi core parallelism.
Therefore the system can reach production oriented throughput without a two orders of magnitude slowdown.

\subsection{Production oriented specifications}

We specify four production mechanisms that are required for deployment.

\paragraph{Persistent audit log with hash chaining.}
Every committed effect records a structured audit event that includes the request hash, the reconstructed commit tag, and a hash pointer to the previous event.
This provides tamper evident auditability.

\paragraph{Key rotation with explicit key identifiers.}
Each policy server supports a set of MAC keys indexed by a key identifier field.
The executor accepts any recognized key identifier and can revoke old keys by removing them from its key set.

\paragraph{Session binding and revocation.}
Every transaction identifier and every sealed handle is bound to a session and caller identity.
Revocation is implemented by invalidating active transaction identifiers and rotating MAC keys.

\paragraph{Crash recovery and replay semantics.}
Replay protection can be backed by a persistent store so that replays across executor restarts are rejected within a configured replay window.

\section{Limitations, failure modes, and responsible disclosure}
\label{sec:limits}

\subsection{Known bypass conditions}

The system does not protect against compromises of the trusted gateway or executor.
The system does not protect against collusion of both policy servers.
The system does not protect against operating system sandbox escapes that violate the capsule mediation contract.

\subsection{False positives and utility risks}

False positives arise from policy databases and from conservative declassification rules.
In the artifact the main sources are recipient allowlist misses, DLP token hits, and leakage budget exhaustion.
We provide per reason code breakdowns to separate policy errors from implementation errors.

\subsection{Channels not fully covered by official benchmarks}

Official benchmarks cover a subset of leakage channels.
We provide a synthetic channel suite for channels that lack official cases.
For paper claims we clearly separate official coverage from synthetic coverage and report them independently.

\subsection{Capsule mediation contract as a verifiable specification}

Skill confined safety is conditional on a capsule mediation contract.
The contract must enforce the following properties.

\paragraph{Minimal filesystem exposure.}
The capsule exposes only the skill code directory and a dedicated write only workspace directory.
Host secret directories are not mounted.

\paragraph{Constrained transport.}
The capsule permits only a single transport endpoint to the gateway, preferably a Unix domain socket.

\paragraph{Network policy.}
The capsule denies public network access.
If the transport uses a network stack then only the gateway address is permitted.

\paragraph{No arbitrary process execution.}
The capsule denies process creation except for a fixed allowlist required by the runtime.

The artifact provides a capsule smoke test that attempts to violate these properties and asserts denial.

\section{Related work and positioning}
\label{sec:related}

SecureClaw combines four themes and its contribution is the system level composition.

\paragraph{Reference monitors and effect boundaries.}
The executor functions as a reference monitor for external side effects.
Unlike prompt based guardrails, the executor enforcement line is non bypassable because it validates cryptographic authorization tied to the exact request.

\paragraph{Privacy preserving policy evaluation.}
Two server PIR and two party computation have been used to hide query contents from a single server.
Our contribution is to apply these primitives to agentic tool use with a fixed shape compilation and a unified routing layer that reduces intent and database selection leakage.

\paragraph{Leakage aware traffic shaping.}
Prior shaping systems often target network observers or focus on padding sizes alone.
Our setting includes an application level policy server that sees structured transcripts.
We therefore define an explicit leakage function and compile the policy check path to a fixed shape schedule that makes the transcript simulatable from that leakage.

\paragraph{Sandboxing and confinement.}
Operating system sandboxes restrict runtime capabilities but do not enforce semantic policy or prevent exfiltration through allowed channels.
SecureClaw treats sandboxing as an optional confinement layer and places semantic enforcement at the gateway and executor boundaries.

\paragraph{Platform native safety layers.}
Native agent platforms provide instruction hierarchy, refusal behavior, and output filtering.
These mechanisms improve safety but remain advisory in a compromised runtime threat model.
SecureClaw provides a separate enforcement line that remains effective even when the runtime intentionally attempts to leak, subject to the trusted gateway and executor assumptions.

\paragraph{Positioning relative to policy only and single server policy.}
A policy only design performs checks but cannot prevent bypass by a runtime that directly invokes tools or replays a previously authorized action.
A single server policy design can enforce non bypassable effects but exposes sensitive policy queries to that server.
SecureClaw addresses both issues by combining a non bypassable executor boundary with two server privacy preserving policy evaluation.

\paragraph{Positioning relative to supply chain scanning.}
Static scanners can flag malicious skills but do not provide an integrity boundary once a skill is installed and executing.
We treat skill ingress as an effectful action that must pass the same preview and commit transaction and the same dual proof executor enforcement.


\bibliographystyle{plainnat}
\bibliography{references}

\appendix
% Appendix-grade proof chain for MIRAGE-OG++ (artifact-faithful).
%
% This file is intended to be \input{} from neurips_2025.tex. It is also
% readable standalone as a LaTeX appendix fragment.

\section{Appendix: Proof Chain for Non-Bypassable Effects and Composition}
\label{app:proofs}

This appendix formalizes MIRAGE-OG++'s core integrity claim---\emph{no external side effect is
committed without two independently authenticated commit proofs bound to the exact request}---and
sketches a compositional argument with the privacy and confinement components. The statements match
the artifact implementation (\texttt{executor\_server/server.py} and \texttt{policy\_server/server.py})
and its formal harness (\texttt{scripts/security\_game\_nbe\_check.py}).

\paragraph{Scope and assumptions.}
We reason about a PPT adversary that fully controls the untrusted agent runtime and all skill code,
and may send arbitrary requests to the gateway and the executor. The executor's MAC verification keys
are not compromised. Policy servers are assumed \emph{honest for integrity} (they compute commit
proofs according to the protocol) and \emph{honest-but-curious for privacy} (single-server leakage is
bounded by an explicit leakage function; see \texttt{LEAKAGE\_MODEL.md}).

\subsection{Cryptographic primitives and canonical binding}
\label{app:proofs:primitives}

\paragraph{MAC.}
Let $\mathsf{MAC} = (\mathsf{Tag},\mathsf{Vrfy})$ be a message authentication code with key space
$\mathcal{K}$ and message space $\mathcal{M}$. We assume existential unforgeability under chosen
message attack (EUF-CMA). The artifact uses HMAC-SHA256:
\[
\mathsf{Tag}_K(m) := \mathrm{HMAC\text{-}SHA256}(K, m), \quad \mathsf{Vrfy}_K(m,t) := [t = \mathsf{Tag}_K(m)].
\]

\paragraph{Canonical serialization and request-binding hash.}
Let $\mathsf{Canon}(\cdot)$ be a deterministic canonical JSON serialization (artifact:
\texttt{common/canonical.py}). Define the request-binding hash:
\[
\mathsf{ReqHash}(\mathsf{intent\_id},\mathsf{caller},\mathsf{session},\mathsf{inputs}) :=
\mathrm{SHA256}\!\left(\mathsf{Canon}\!\left(
\begin{array}{l}
\{\texttt{"intent\_id"}:\mathsf{intent\_id},\ \texttt{"caller"}:\mathsf{caller},\ \texttt{"session"}:\mathsf{session},\\
\ \texttt{"inputs"}:\mathsf{inputs}\}
\end{array}\right)\right).
\]
We assume SHA-256 collision resistance and \emph{binding hash consistency}: both gateway and executor
compute identical byte strings under $\mathsf{Canon}(\cdot)$ for semantically equal JSON objects.
This is an engineering assumption enforced by using the same library routine in both places.

\subsection{Commit proof syntax and executor acceptance predicate}
\label{app:proofs:accept}

Each policy server $\mathcal{P}_\sigma$ ($\sigma\in\{0,1\}$) holds a MAC key $K_\sigma$ (with a key id
\texttt{kid} for rotation) and issues \emph{commit proof shares}. A commit proof share is a pair
$\pi_\sigma = (\mathsf{meta}_\sigma, t_\sigma)$, where:
\begin{align*}
\mathsf{meta}_\sigma := \{&
\texttt{"v"}:1,\ \texttt{"kind"}:\texttt{"commit"},\ \texttt{"server\_id"}:\sigma,\ \texttt{"kid"}:\mathrm{kid},\ \texttt{"ts"}:\tau,\\
&\texttt{"action\_id"}:a,\ \texttt{"program\_id"}:p,\ \texttt{"request\_sha256"}:h,\\
&\texttt{"outputs"}:y_\sigma,\ \texttt{"commit\_tag\_share"}:\mathrm{tag}_\sigma\},
\end{align*}
and $t_\sigma := \mathsf{Tag}_{K_\sigma}(\mathsf{Canon}(\mathsf{meta}_\sigma))$.
The artifact computes and verifies these fields in:
\texttt{policy\_server/server.py:/mpc/finalize} and
\texttt{executor\_server/server.py:\_verify\_commit\_evidence}.

\paragraph{Executor acceptance.}
Given an effect request $\mathsf{req}$ and two purported commit proof shares $\pi_0,\pi_1$, the
executor $\mathcal{X}$ accepts if and only if the predicate $\mathsf{Accept}(\mathsf{req},\pi_0,\pi_1)$
holds, where:
\begin{enumerate}
\item Both shares parse, have \texttt{v=1} and \texttt{kind="commit"}, and claim distinct
\texttt{server\_id} values in $\{0,1\}$.
\item MAC verification succeeds:
$\mathsf{Vrfy}_{K_\sigma}(\mathsf{Canon}(\mathsf{meta}_\sigma), t_\sigma)=1$ for $\sigma\in\{0,1\}$,
using the key for the claimed \texttt{kid}.
\item Freshness: $\lvert \mathrm{now} - \tau\rvert \leq \mathrm{TTL}_{\mathsf{mac}}$
(artifact env \texttt{POLICY\_MAC\_TTL\_S}).
\item Context agreement: both shares bind the same $(a,p,h)$ tuple.
\item Request binding: letting $(\mathsf{intent\_id},\mathsf{caller},\mathsf{session},\mathsf{inputs})$
be the effectful fields of $\mathsf{req}$, we require $h =
\mathsf{ReqHash}(\mathsf{intent\_id},\mathsf{caller},\mathsf{session},\mathsf{inputs})$.
\item Replay protection: $(a,p,h)$ (equivalently $a$) has not been accepted before within the replay
window (artifact: \texttt{\_ReplayGuard.check\_and\_mark}).
\item Policy bits reconstructed from $(y_0,y_1)$ indicate allow; if a confirmation bit is set then
the request includes explicit user confirmation; and any sanitize patch is applied.
\end{enumerate}

\subsection{Security game for non-bypassable effects}
\label{app:proofs:nbe-game}

We formalize \emph{non-bypassable effects} as an integrity game where the adversary controls the
untrusted runtime and can obtain valid commit proofs for adaptively chosen preview tuples, but must
not be able to commit an effect outside the authorized set.

\paragraph{Oracles.}
The challenger provides a \emph{commit oracle} $\mathcal{O}_{\mathsf{commit}}(\rho)$ that, on input
$\rho = (\mathsf{intent\_id},\mathsf{caller},\mathsf{session},\mathsf{inputs})$, returns a fresh
action id $a$ and valid proof shares $(\pi_0,\pi_1)$ whose \texttt{request\_sha256} equals
$\mathsf{ReqHash}(\rho)$. The challenger records the authorized binding tuple
$(a,p,\mathsf{ReqHash}(\rho),\rho)$ in a set $\mathcal{S}$.

\paragraph{Experiment $\mathsf{Exp}^{\mathsf{NBE}}(\lambda)$.}
\begin{enumerate}
\item The challenger samples independent MAC keys $K_0,K_1 \leftarrow \mathcal{K}$ and initializes
replay state empty.
\item The adversary $\mathcal{A}$ may adaptively query $\mathcal{O}_{\mathsf{commit}}(\rho)$.
\item $\mathcal{A}$ outputs an effect request $\mathsf{req}^\star$ and two shares
$(\pi_0^\star,\pi_1^\star)$.
\item $\mathcal{A}$ wins if $\mathsf{Accept}(\mathsf{req}^\star,\pi_0^\star,\pi_1^\star)=1$ and at
least one of the following holds:
\begin{enumerate}
\item \textbf{No-auth:} the accepted $(a^\star,p^\star,h^\star)$ tuple is not in $\mathcal{S}$.
\item \textbf{Binding break:} $(a^\star,p^\star,h^\star)$ is in $\mathcal{S}$, but the request
tuple $\rho(\mathsf{req}^\star)$ differs from the $\rho$ recorded for that context.
\item \textbf{Replay:} the executor accepts the same $a^\star$ twice within the replay window.
\end{enumerate}
\end{enumerate}

\subsection{Main theorem: no side effects without dual proofs}
\label{app:proofs:nbe-thm}

\begin{theorem}[No side effects without dual commit proofs]\label{thm:nbe}
Assume (i) $\mathsf{MAC}$ is EUF-CMA secure; (ii) SHA-256 is collision resistant and
$\mathsf{Canon}(\cdot)$ is consistent; (iii) the executor enforces $\mathsf{Accept}$ as specified and
its replay state does not lose entries within the replay window. Then for any PPT adversary
$\mathcal{A}$ there exist PPT reductions $\mathcal{B}_0,\mathcal{B}_1,\mathcal{C}$ such that:
\[
\Pr[\mathsf{Exp}^{\mathsf{NBE}}(\lambda)=1]
\le \mathrm{Adv}^{\mathrm{euf\text{-}cma}}_{\mathsf{MAC}}(\mathcal{B}_0)
 + \mathrm{Adv}^{\mathrm{euf\text{-}cma}}_{\mathsf{MAC}}(\mathcal{B}_1)
 + \mathrm{Adv}^{\mathrm{coll}}_{\mathrm{SHA256}}(\mathcal{C})
 + \varepsilon_{\mathrm{replay}}(\lambda),
\]
where $\varepsilon_{\mathrm{replay}}$ captures only non-cryptographic replay-store failures excluded
from the model.
\end{theorem}

\paragraph{Proof (reduction by cases).}
Fix an adversary $\mathcal{A}$ that wins the experiment and consider the accepting transcript
$(\mathsf{req}^\star,\pi_0^\star,\pi_1^\star)$.
By acceptance, both MACs verify and both meta records bind the same context
$(a^\star,p^\star,h^\star)$ with fresh timestamps, and $h^\star$ equals the executor's recomputed
$\mathsf{ReqHash}(\rho(\mathsf{req}^\star))$.

\emph{Case 1 (No-auth).} Suppose $(a^\star,p^\star,h^\star)\not\in\mathcal{S}$. The challenger never
queried its own commit oracle on the message $\mathsf{Canon}(\mathsf{meta}_0^\star)$ or
$\mathsf{Canon}(\mathsf{meta}_1^\star)$ for this context. Since both tags verify, at least one share
is a fresh valid tag on a message not previously tagged under the corresponding key. We build a UF-CMA
forger $\mathcal{B}_0$ (symmetrically $\mathcal{B}_1$) that embeds its UF-CMA challenge key as $K_0$,
answers $\mathcal{O}_{\mathsf{commit}}$ by querying its signing oracle for $\mathsf{Canon}(\mathsf{meta}_0)$,
and outputs $(\mathsf{Canon}(\mathsf{meta}_0^\star),t_0^\star)$ when $\mathcal{A}$ wins in this case.

\emph{Case 2 (Binding break).} Suppose $(a^\star,p^\star,h^\star)\in\mathcal{S}$, but
$\rho(\mathsf{req}^\star)\neq \rho$ recorded for that context. Acceptance implies:
\[
h^\star = \mathsf{ReqHash}(\rho(\mathsf{req}^\star)) = \mathsf{ReqHash}(\rho).
\]
If $\rho(\mathsf{req}^\star)\neq \rho$ then we have found a collision in SHA-256 \emph{under the
canonical encoding} (or an inconsistency in $\mathsf{Canon}$). A reduction $\mathcal{C}$ outputs the
two distinct canonical inputs whose hashes are equal. This contradicts collision resistance or the
engineering consistency assumption.

\emph{Case 3 (Replay).} If the executor accepts a previously accepted $a^\star$ within the replay
window, then the replay guard's check-and-mark failed due to loss/corruption of replay state or lack
of atomicity. This is outside the cryptographic model and is captured by $\varepsilon_{\mathrm{replay}}$.

Taking a union bound over the cases yields the theorem.
\qed

\subsection{Session, caller, TTL binding}
\label{app:proofs:binding}

Theorem~\ref{thm:nbe} already incorporates binding of $\mathsf{caller}$ and $\mathsf{session}$ because
they are included in $\mathsf{ReqHash}$. Freshness (TTL) is enforced by the executor's timestamp
check. Thus, any attempt to reuse a commit proof for a different caller, session, or effectful input
must either fail MAC verification/binding checks or imply a hash collision.

\subsection{Composition with privacy and confinement}
\label{app:proofs:composition}

We briefly state a modular composition claim that matches the artifact's design.

\begin{theorem}[Modular composition sketch]\label{thm:composition}
Assume:
(i) NBE holds as in Theorem~\ref{thm:nbe};
(ii) secret myopia holds for the handle store with explicit declassification (formalized in
\texttt{FORMAL\_SECURITY.md});
(iii) each policy server transcript is simulatable from an explicit leakage function $L_{\mathsf{policy}}$
(\texttt{LEAKAGE\_MODEL.md}) under the DPF/FSS and MPC security assumptions; and
(iv) a capsule mediation contract $MC$ holds that makes the gateway the unique I/O path for skills.
Then the composed system simultaneously satisfies NBE, SM, SAP$(L_{\mathsf{policy}})$, and conditional
SCS (conditioned on $MC$), with overall failure probability at most the sum of the component failure
probabilities.
\end{theorem}

\paragraph{Discussion.}
The proof is a standard hybrid argument: integrity (NBE) is enforced at the executor boundary; SM is
enforced at the gateway boundary by handleization and explicit declassification; SAP is about the view
of a single policy server and follows from the transcript design and the DPF/MPC security properties
up to leakage; and SCS is conditioned on the capsule contract. Since these properties concern
disjoint attacker views and disjoint enforcement points, a union bound suffices.

% Appendix: leakage functions and mapping to implementation.
% Intended to be \input{} from neurips_2025.tex.

\section{Appendix: Leakage model and implementation mapping}
\label{app:leakage}

This appendix defines the explicit leakage functions used by our single auditor privacy claim and by our end to end channel analysis.
The definitions are artifact faithful and correspond to the implementation in \texttt{LEAKAGE\_MODEL.md}.

\subsection{Notation}

Let $\lambda$ be the security parameter.
Let $\mathcal{G}$ be the gateway, let $\mathcal{P}_0$ and $\mathcal{P}_1$ be the policy servers, and let $\mathcal{X}$ be the executor.
Let an action be identified by an action identifier $a$.
Let the policy program identifier be $p$.
Let the request binding hash be $h := \mathsf{ReqHash}$ as defined in Appendix \ref{app:proofs:primitives}.

We write $\mathsf{View}(\mathcal{P}_\sigma)$ for the full transcript observed by policy server $\mathcal{P}_\sigma$, including request paths, message sizes, coarse timing buckets, and application layer fields.

\subsection{Policy server leakage function $L_{\mathsf{policy}}$}

Single auditor privacy is defined relative to a leakage function that captures the intentionally revealed metadata to one policy server.
We define
\[
L_{\mathsf{policy}} := \langle L_{\mathsf{pir}},\ L_{\mathsf{mpc}},\ L_{\mathsf{confirm}},\ L_{\mathsf{time}} \rangle .
\]

\subsubsection{PIR leakage $L_{\mathsf{pir}}$}

The gateway queries membership databases using two server PIR based on DPF and FSS.
A single policy server observes a key share per query and cannot infer the queried index under the DPF security assumption.

We capture the remaining metadata leakage as follows.

\paragraph{Public configuration.}
The domain size $N$ and the fixed number of keys per subrequest are public.
Bundling configuration is public.

\paragraph{Constant shape schedule.}
If mixing is enabled then the gateway sends exactly $\mathsf{pad\_to}$ subrequests per tick and may fill slack with dummy subrequests.
A single policy server learns the tick index and the fixed number of subrequests per tick.

\paragraph{Database name exposure.}
If bundling is enabled then the policy server sees a single bundle database name for all membership checks.
Otherwise the policy server sees the database name per request.

Therefore we define $L_{\mathsf{pir}}$ as the tuple containing
the public domain size,
the fixed keys per subrequest,
the observed database name,
the tick schedule parameters,
and the multiset of action identifiers included in each tick.

\subsubsection{Two party computation leakage $L_{\mathsf{mpc}}$}

The policy program is evaluated using two party computation over XOR shared inputs.
We assume honest but curious privacy for the two party computation protocol.
In unified mode the circuit and program identifier are fixed and do not depend on the intent category.

We define $L_{\mathsf{mpc}}$ as the tuple containing
the program identifier,
the circuit shape metadata which is constant in unified mode,
the tick schedule parameters,
and the multiset of action identifiers included in each tick.

\subsubsection{Confirm path leakage $L_{\mathsf{confirm}}$}

When confirm path scanning is enabled the gateway may execute block PIR queries for a deterministic finite automaton transition table.
If the confirm path is invoked conditionally then a single policy server may learn whether a scan occurred.
If step padding is disabled then a single policy server may learn the number of steps.

We define $L_{\mathsf{confirm}}$ to include
the scan invocation bit,
the number of scan steps after any configured padding,
the public block size,
and the public transition table parameters.

\subsubsection{Timing leakage $L_{\mathsf{time}}$}

The policy server can observe coarse arrival times and response times.
We do not model kernel side channels.
We define a quantized timing bucket function and include only the bucket index per tick.
This is enforced by fixed interval mixing when enabled.
Concretely, if the tick interval is $\Delta$ then the policy server learns at most the number of
ticks in a window and the bucketed offset within each interval.

\subsection{System leakage function $L_{\mathsf{sys}}$}

We define application level leakage across seven channels inspired by AgentLeak.
For an execution trace we define
\[
L_{\mathsf{sys}} := \langle L_{C1}, L_{C2}, L_{C3}, L_{C4}, L_{C5}, L_{C6}, L_{C7} \rangle .
\]

\subsubsection{Channel definitions}

\paragraph{$C1$ final output.}
$L_{C1}$ includes the final decision type, a coarse output length, a reason code, and a timing bucket.
It excludes sensitive plaintext unless explicitly declassified.

\paragraph{$C2$ inter agent messages.}
$L_{C2}$ includes message counts, coarse timing, and handle identifiers used as payload references.
It excludes sensitive plaintext because payloads are handleized.

\paragraph{$C3$ tool inputs.}
$L_{C3}$ includes the intent identifier or a shadow class identifier, the set of input key names, a reason code, and a timing bucket.
It excludes sensitive plaintext because sensitive fields are represented as handles.

\paragraph{$C4$ tool outputs.}
$L_{C4}$ includes a tool status code, a coarse output summary length, handle identifiers returned as artifacts, and a timing bucket.
It excludes sensitive plaintext unless explicitly declassified.

\paragraph{$C5$ memory.}
$L_{C5}$ includes the memory namespace, a key identifier, a record count, handle identifiers stored, and a timing bucket.
It excludes plaintext memory values because memory stores handle records.

\paragraph{$C6$ logs and audit.}
$L_{C6}$ includes audit event type, caller identity, session identity, reason codes, and hash chain metadata.
It excludes plaintext secrets.

\paragraph{$C7$ skill supply chain ingress.}
$L_{C7}$ includes a skill digest, membership hit bits for known bad indicators, an install marker bit, a reason code, and a timing bucket.
It excludes raw skill content unless explicitly declassified.

\subsubsection{Channel to implementation mapping}

Table \ref{tab:channel-map} maps each channel leakage function to the implementation path and primary controls.

\begin{table}[t]
\centering
\small
\begin{tabular}{lll}
\toprule
Channel & Primary implementation path & Primary controls \\
\midrule
$C1$ & \texttt{gateway/mcp\_server.py}, \texttt{gateway/egress\_policy.py} & reason codes, sanitize patch, leakage budget \\
$C2$ & \texttt{gateway/router.py}, \texttt{gateway/handles.py} & handle bindings, leakage budget \\
$C3$ & \texttt{gateway/mcp\_server.py}, \texttt{gateway/policy\_unified.py} & unified routing, bundle selection \\
$C4$ & \texttt{gateway/executors/*} & output handleization, declassification \\
$C5$ & \texttt{gateway/executors/memoryexec.py} & memory handle mediation, leakage budget \\
$C6$ & \texttt{gateway/audit.py}, \texttt{executor\_server/server.py} & structured metadata logging, hash chain \\
$C7$ & \texttt{gateway/policy\_unified.py}, \texttt{gateway/capabilities.yaml} & skill IOC checks, caller capability policy \\
\bottomrule
\end{tabular}
\caption{Per-channel leakage mapping to implementation and controls.}
\label{tab:channel-map}
\end{table}

\subsection{Residual leakage and allowed leakage items}

We list the allowed leakage items and their sources.
Each item is either required for correctness or explicitly accepted as leakage.

\begin{itemize}
\item Action identifier exposure to policy servers through fixed shape batching.
\item Program identifier exposure to policy servers through unified routing.
\item Tick schedule parameters such as interval and pad size.
\item Coarse timing buckets at tick granularity.
\item Audit metadata such as reason codes and hash chain pointers.
\end{itemize}

We treat the following as residual risks that are not eliminated by default configuration.

\begin{itemize}
\item Conditional confirm path invocation may leak a branch bit.
\item Low entropy guessing attacks against unkeyed request hashes are possible in principle when an attacker can enumerate candidates.
\end{itemize}

\subsection{Mapping to implementation and configuration}

Table \ref{tab:leakmap} maps the major leakage controls to code paths and configuration keys.

\begin{table}[t]
\centering
\small
\begin{tabular}{lll}
\toprule
Control & Code path & Configuration \\
\midrule
Request binding hash & \texttt{common/canonical.py} & none \\
Commit proof verify & \texttt{executor\_server/server.py} & \texttt{POLICY\_MAC\_TTL\_S} \\
Unified routing and program id & \texttt{gateway/policy\_unified.py} & \texttt{UNIFIED\_POLICY}, \texttt{MIRAGE\_POLICY\_PROGRAM\_ID} \\
PIR bundling & \texttt{gateway/fss\_pir.py} & \texttt{USE\_POLICY\_BUNDLE} \\
Oblivious bundle selection & \texttt{policy\_server/build\_dbs.py} & \texttt{POLICY\_BUNDLE\_NAME} \\
PIR mixing and padding & \texttt{gateway/fss\_pir.py} & \texttt{PIR\_MIX\_ENABLED}, \texttt{PIR\_MIX\_PAD\_TO} \\
PIR cover traffic & \texttt{gateway/fss\_pir.py} & \texttt{PIR\_COVER\_TRAFFIC} \\
MPC mixing and padding & \texttt{gateway/policy\_unified.py} & \texttt{MPC\_MIX\_ENABLED}, \texttt{MPC\_MIX\_PAD\_TO} \\
MPC cover traffic & \texttt{gateway/policy\_unified.py} & \texttt{MPC\_COVER\_TRAFFIC} \\
Confirm path DFA scan & \texttt{gateway/guardrails.py} & \texttt{DLP\_MODE}, \texttt{PAD\_DFA\_STEPS} \\
Leakage budgets & \texttt{gateway/leakage\_budget.py} & \texttt{LEAKAGE\_BUDGET\_*} \\
\bottomrule
\end{tabular}
\caption{Leakage controls and their implementation mapping.}
\label{tab:leakmap}
\end{table}

\end{document}
